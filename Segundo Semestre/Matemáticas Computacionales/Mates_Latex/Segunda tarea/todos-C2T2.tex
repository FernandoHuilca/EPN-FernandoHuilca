% Options for packages loaded elsewhere
\PassOptionsToPackage{unicode}{hyperref}
\PassOptionsToPackage{hyphens}{url}
\PassOptionsToPackage{dvipsnames,svgnames,x11names}{xcolor}
%
\documentclass[
  letterpaper,
  DIV=11,
  numbers=noendperiod]{scrartcl}

\usepackage{amsmath,amssymb}
\usepackage{iftex}
\ifPDFTeX
  \usepackage[T1]{fontenc}
  \usepackage[utf8]{inputenc}
  \usepackage{textcomp} % provide euro and other symbols
\else % if luatex or xetex
  \usepackage{unicode-math}
  \defaultfontfeatures{Scale=MatchLowercase}
  \defaultfontfeatures[\rmfamily]{Ligatures=TeX,Scale=1}
\fi
\usepackage{lmodern}
\ifPDFTeX\else  
    % xetex/luatex font selection
\fi
% Use upquote if available, for straight quotes in verbatim environments
\IfFileExists{upquote.sty}{\usepackage{upquote}}{}
\IfFileExists{microtype.sty}{% use microtype if available
  \usepackage[]{microtype}
  \UseMicrotypeSet[protrusion]{basicmath} % disable protrusion for tt fonts
}{}
\makeatletter
\@ifundefined{KOMAClassName}{% if non-KOMA class
  \IfFileExists{parskip.sty}{%
    \usepackage{parskip}
  }{% else
    \setlength{\parindent}{0pt}
    \setlength{\parskip}{6pt plus 2pt minus 1pt}}
}{% if KOMA class
  \KOMAoptions{parskip=half}}
\makeatother
\usepackage{xcolor}
\usepackage[top=30mm,left=30mm]{geometry}
\setlength{\emergencystretch}{3em} % prevent overfull lines
\setcounter{secnumdepth}{5}
% Make \paragraph and \subparagraph free-standing
\ifx\paragraph\undefined\else
  \let\oldparagraph\paragraph
  \renewcommand{\paragraph}[1]{\oldparagraph{#1}\mbox{}}
\fi
\ifx\subparagraph\undefined\else
  \let\oldsubparagraph\subparagraph
  \renewcommand{\subparagraph}[1]{\oldsubparagraph{#1}\mbox{}}
\fi


\providecommand{\tightlist}{%
  \setlength{\itemsep}{0pt}\setlength{\parskip}{0pt}}\usepackage{longtable,booktabs,array}
\usepackage{calc} % for calculating minipage widths
% Correct order of tables after \paragraph or \subparagraph
\usepackage{etoolbox}
\makeatletter
\patchcmd\longtable{\par}{\if@noskipsec\mbox{}\fi\par}{}{}
\makeatother
% Allow footnotes in longtable head/foot
\IfFileExists{footnotehyper.sty}{\usepackage{footnotehyper}}{\usepackage{footnote}}
\makesavenoteenv{longtable}
\usepackage{graphicx}
\makeatletter
\def\maxwidth{\ifdim\Gin@nat@width>\linewidth\linewidth\else\Gin@nat@width\fi}
\def\maxheight{\ifdim\Gin@nat@height>\textheight\textheight\else\Gin@nat@height\fi}
\makeatother
% Scale images if necessary, so that they will not overflow the page
% margins by default, and it is still possible to overwrite the defaults
% using explicit options in \includegraphics[width, height, ...]{}
\setkeys{Gin}{width=\maxwidth,height=\maxheight,keepaspectratio}
% Set default figure placement to htbp
\makeatletter
\def\fps@figure{htbp}
\makeatother

\KOMAoption{captions}{tableheading}
\makeatletter
\makeatother
\makeatletter
\makeatother
\makeatletter
\@ifpackageloaded{caption}{}{\usepackage{caption}}
\AtBeginDocument{%
\ifdefined\contentsname
  \renewcommand*\contentsname{Table of contents}
\else
  \newcommand\contentsname{Table of contents}
\fi
\ifdefined\listfigurename
  \renewcommand*\listfigurename{List of Figures}
\else
  \newcommand\listfigurename{List of Figures}
\fi
\ifdefined\listtablename
  \renewcommand*\listtablename{List of Tables}
\else
  \newcommand\listtablename{List of Tables}
\fi
\ifdefined\figurename
  \renewcommand*\figurename{Figure}
\else
  \newcommand\figurename{Figure}
\fi
\ifdefined\tablename
  \renewcommand*\tablename{Table}
\else
  \newcommand\tablename{Table}
\fi
}
\@ifpackageloaded{float}{}{\usepackage{float}}
\floatstyle{ruled}
\@ifundefined{c@chapter}{\newfloat{codelisting}{h}{lop}}{\newfloat{codelisting}{h}{lop}[chapter]}
\floatname{codelisting}{Listing}
\newcommand*\listoflistings{\listof{codelisting}{List of Listings}}
\makeatother
\makeatletter
\@ifpackageloaded{caption}{}{\usepackage{caption}}
\@ifpackageloaded{subcaption}{}{\usepackage{subcaption}}
\makeatother
\makeatletter
\makeatother
\ifLuaTeX
  \usepackage{selnolig}  % disable illegal ligatures
\fi
\IfFileExists{bookmark.sty}{\usepackage{bookmark}}{\usepackage{hyperref}}
\IfFileExists{xurl.sty}{\usepackage{xurl}}{} % add URL line breaks if available
\urlstyle{same} % disable monospaced font for URLs
\hypersetup{
  pdftitle={EJERCICIOS C2T2},
  pdfauthor={Betancourt Alison; Angulo Javier; Anrango Stalin; Huilca Fernando; Sarasti Sebastian; Simbaña Mateo},
  colorlinks=true,
  linkcolor={blue},
  filecolor={Maroon},
  citecolor={Blue},
  urlcolor={Blue},
  pdfcreator={LaTeX via pandoc}}

\title{EJERCICIOS C2T2}
\author{Betancourt Alison \and Angulo Javier \and Anrango
Stalin \and Huilca Fernando \and Sarasti Sebastian \and Simbaña Mateo}
\date{}

\begin{document}
\maketitle
\renewcommand*\contentsname{Contenido}
{
\hypersetup{linkcolor=}
\setcounter{tocdepth}{3}
\tableofcontents
}
\hypertarget{ejercicio-1}{%
\section{EJERCICIO 1}\label{ejercicio-1}}

Encuentre \(S_{1}\),donde \(S\) es una sucesión definida como:

\begin{itemize}
\tightlist
\item
  \(c, d, d, c, d, c\)
\end{itemize}

R ) c

\hypertarget{ejercicio-7}{%
\section{EJERCICIO 7}\label{ejercicio-7}}

Dada la fórmula \(t_n = 2n - 1\), encontrar el valor de \(t_{2077}\).

Resolución:

\begin{enumerate}
\def\labelenumi{\arabic{enumi}.}
\tightlist
\item
  Sustituimos \(n = 2077\) en la fórmula:

  \begin{center}
    $ t_{2077} = 2 \times 2077 - 1$
    \end{center}
\item
  Calculamos:
\end{enumerate}

\begin{center}
  $t_{2077} = 4154 - 1$
  
  $t_{2077} = 4153$
  \end{center}

\hypertarget{ejercicio-10}{%
\section{EJERCICIO 10}\label{ejercicio-10}}

Responda a los ejercicios 4 al 16 para la sucesión t definida por:

\[t_n= 2n − 1, n ≥ 1\]

Encuentre \[ \prod_{i=1}^{3} t_i = 1*3*5 = 15\].

Para \(i = 1\), \(t_1 = 2(1)-1 = 1\)

Para \(i = 2\), \(t_2 = 2(2)-1 = 3\)

Para \(i = 3\), \(t_3 = 2(3)-1 = 5\)

\hypertarget{ejercicio-21}{%
\section{EJERCICIO 21}\label{ejercicio-21}}

Responda a los ejercicios 17 al 24 para la sucesión (v) definida por \[
\begin{align*}
V_n = n! + 2, \quad n \geq 1.
\end{align*}
\] 21. ¿Es (v) creciente?

Para determinar si la función es creciente, podemos analizar la
derivada:

\hfill\break
\begin{align*}
V'_n &= \frac{d}{dn} (n! + 2) \\
V'_n &= n! \cdot \ln(n)
\end{align*}\\

Ahora evaluamos la derivada en (n = 1) y (n = 2):

\hfill\break
\begin{align*}
V'_1 &= 1! \cdot \ln(1) \\
V'_1 &= 0 \\
V'_2 &= 2! \cdot \ln(2)
\end{align*}\\

Para (n = 2), los valores siempre son positivos; por lo tanto, la
función es creciente para (n \geq 2).

Esto también se puede comprobar con la gráfica:

\begin{figure}

{\centering \includegraphics{todos C2T2_files/mediabag/uUDPgdt.pdf}

}

\caption{Diagrama de la función v\_n}

\end{figure}

\hypertarget{ejercicio-24}{%
\section{EJERCICIO 24}\label{ejercicio-24}}

\textbf{Responda a los ejercicios 17 al 24 para la sucesión V definida
por \(V_{n} = n! + 2, \ n \geq 1\).} ¿Es V no decreciente?

Para poder determinar si es no decreciente, hay que plasmar la sucesión.

\(V_{1} = 1! + 2 \ = 3\)

\(V_{2} = 2! + 2 \ = 4\)

\(V_{3} = 3! + 2 \ = 8\)

\(V_{4} = 4! + 2 \ = 26\)

La sucesión es: 3, 4, 8, 26, \ldots{}

Para determinar si es \emph{no decreciente} se establece:

\{=tex\}

\begin{center}
$S_{n} \leq S_{n+1}$
\end{center}

\(¿V_{1} \leq V_{2}? \rightarrow \ VERDADERO\)

\(¿V_{2} \leq V_{3}? \rightarrow \ VERDADERO\)

\(¿V_{3} \leq V_{4}? \rightarrow \ VERDADERO\)

También se puede denotar que los valores de V siempre van aumentando,
por lo tanto, \emph{V si es NO decreciente}.

\hypertarget{ejercicio-40}{%
\section{EJERCICIO 40}\label{ejercicio-40}}

\emph{Dada la sucesión definida por:}

\(a_{n} = n^{2} - 3n + 3 , n \geq 1\)

Encuentre \(\displaystyle\sum_{j=3}^{5} a_i\)

\(\displaystyle\sum_{j=3}^{5} a_i = (3^{2}-3(3)+3) +(4^{2}-3(4)+3) + (5^{2}-3(5)+3)\)

\(\displaystyle\sum_{j=3}^{5} a_i = (9-9+3) + (16-12+3)+(25-15+3)\)

\(\displaystyle\sum_{j=3}^{5} a_i = 3 + 7 + 13\)

\(\displaystyle\sum_{j=3}^{5} a_i = 23\)



\end{document}
